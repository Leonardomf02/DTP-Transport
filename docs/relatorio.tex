\documentclass[11pt,a4paper]{article}
\usepackage[margin=2.2cm]{geometry}
\usepackage{times}
\usepackage{titlesec}
\usepackage{enumitem}
\usepackage{hyperref}
\usepackage{graphicx}
\usepackage{fancyhdr}
\usepackage{setspace}
\usepackage{listings}

\setlength{\parskip}{6pt}
\setlength{\parindent}{0pt}

\titleformat{\section}{\large\bfseries}{\thesection}{1em}{}

\title{\textbf{DTP — Deadline-aware Transport Protocol}\\
\large Arquiteturas Avançadas de Redes — UBI (2025/26)}
\author{
Leonardo Santos (M15811) \\
Diogo Pedro (E11657)\\
Guilherme Vicente (E11656)\\
Tiago Ramos (E11663)
}

\date{}

\begin{document}
\maketitle

\begin{abstract}
O DTP (Deadline-aware Transport Protocol) é um protocolo de transporte proposto para aplicações interativas e sensíveis ao tempo, onde a entrega antes de um prazo (deadline) é mais importante que a fiabilidade absoluta. Este relatório descreve a motivação, o problema identificado, uma revisão de literatura relevante, o mini-RFC que define o funcionamento do DTP, o seu modelo de operação, formato de mensagens e políticas de scheduling. A secção de implementação e resultados será preenchida posteriormente com o protótipo a desenvolver. 
\end{abstract}

\section{Introdução}

Aplicações modernas como VR/AR, sistemas de telemedicina, controlo robótico e streaming interativo exigem requisitos temporais estritos. Protocolos tradicionais como TCP priorizam fiabilidade e congestion control, mas ignoram deadlines; pacotes que chegam tarde perdem utilidade.

O DTP (Deadline-aware Transport Protocol) aborda este problema ao permitir que aplicações definam explicitamente deadlines para cada unidade de dados. O protocolo gere prioridades, descarta informação expirada e tenta maximizar a fração de pacotes entregues antes do deadline.

\section{Revisão de Literatura}

\textbf{Deadline-aware networking.} Trabalhos recentes como Shi et al. (APNet 2019) discutem transportes que incorporam deadlines explícitas, principalmente em datacenters. A ideia é aproximar o comportamento do transporte às necessidades reais de aplicações interativas.

\textbf{QUIC e transporte moderno.} QUIC oferece latências baixas e streams multiplexados, mas não possui um mecanismo nativo para deadlines. Extensões propostas exploram prioridades, mas continuam centradas em fiabilidade.

\textbf{AQM e redução de latência.} Técnicas como CoDel e FQ-CoDel lidam com filas e latência ao nível da rede, mas não utilizam informação temporal fornecida pela aplicação.

\textbf{FEC.} RFC 6363 descreve mecanismos de FEC aplicados a fluxos multimédia para reduzir impacto de perdas sem retransmissões. No contexto de mensagens com deadlines, FEC leve pode ser útil.

\textbf{Pontos-chave da literatura:}
\begin{itemize}[noitemsep]
    \item Os protocolos existentes não incorporam deadlines explícitas.
    \item Aplicações implementam o seu próprio controlo temporal, criando inconsistência.
    \item Prioridades existem, mas não refletem urgência temporal.
\end{itemize}

\section{Problema Identificado}

O problema em aberto prende-se com a incapacidade dos protocolos de transporte tradicionais em distinguir dados que deixam de ser úteis após um prazo.

\subsection*{Limitações das abordagens atuais}
\begin{itemize}[noitemsep]
    \item TCP retrasa ou retransmite pacotes que já perderam utilidade.
    \item UDP não fornece qualquer mecanismo de scheduling ou priorização.
    \item Soluções de aplicação (e.g., descarte manual de frames) duplicam esforços.
\end{itemize}

\subsection*{Implicações práticas}
\begin{itemize}[noitemsep]
    \item Experiência degradada em VR e videoconferência.
    \item Atrasos acumulados devido a filas sem noção de deadlines.
    \item Desperdício de largura de banda em retransmissões desnecessárias.
\end{itemize}

O DTP procura integrar deadlines no próprio transporte para corrigir estas limitações.

\section{Proposta de Protocolo (Mini-RFC)}

\subsection{Objetivo}
Fornecer um protocolo leve, baseado em UDP, que permita ao emissor enviar pacotes com deadlines explícitas, prioridades e informação temporal suficiente para que o receptor e a rede possam tomar decisões orientadas à utilidade temporal dos dados.

\subsection{Arquitetura Geral}

\textbf{Camada:} Transporte (sobre UDP).  
\textbf{Entidades:} Sender, Receiver e Proxy opcional.

Fluxo: Aplicação $\rightarrow$ DTP Sender $\rightarrow$ UDP $\rightarrow$ DTP Receiver $\rightarrow$ Aplicação.

O relógio pode ser local ou relativo, com deadlines especificadas em milissegundos após envio.

\subsection{Formato das Mensagens}

\begin{verbatim}
0                   1                   2                   3
0 1 2 3 4 5 6 7 8 9 0 1 2 3 4 5 6 7 8 9 0 1 2 3 4 5 6 7 8 9 0 1
+-+-+-+-+-+-+-+-+-+-+-+-+-+-+-+-+-+-+-+-+-+-+-+-+-+-+-+-+-+-+-+-+
| Ver |Flags|MsgType|Pri| StreamID |         Seq Number         |
+-+-+-+-+-+-+-+-+-+-+-+-+-+-+-+-+-+-+-+-+-+-+-+-+-+-+-+-+-+-+-+-+
|                       Deadline_ms (32 bits)                    |
+-+-+-+-+-+-+-+-+-+-+-+-+-+-+-+-+-+-+-+-+-+-+-+-+-+-+-+-+-+-+-+-+
|                     Timestamp_send (32 bits)                   |
+-+-+-+-+-+-+-+-+-+-+-+-+-+-+-+-+-+-+-+-+-+-+-+-+-+-+-+-+-+-+-+-+
|    PayloadLen   | Reserved  |       Payload ...                |
+-+-+-+-+-+-+-+-+-+-+-+-+-+-+-+-+-+-+-+-+-+-+-+-+-+-+-+-+-+-+-+-+
\end{verbatim}

\subsection{Tipos de Mensagens}

\begin{itemize}[noitemsep]
    \item DATA — dados com deadline.
    \item ACK/NACK — confirmação ou indicação de perda.
    \item SYN/FIN — início e fecho de sessão.
    \item CTRL — sinalização opcional de métricas.
\end{itemize}

\subsection{Operações e Estados}

\textbf{Sender:}
\begin{itemize}[noitemsep]
    \item IDLE $\rightarrow$ SENDING
    \item SENDING $\rightarrow$ BACKPRESSURE (fila cheia)
    \item SENDING $\rightarrow$ CLOSED
\end{itemize}

\textbf{Receiver:} LISTEN → RECEIVING → CLOSED.

\subsection{Scheduling e Congestion Control}

\textbf{Scheduler principal:} EDF (Earliest Deadline First), com aging para evitar starvation.

\textbf{Pacing:} baseado em estimativas de RTT e perda, ajustando taxa de envio.

\textbf{Política de Descarte:} pacotes expirados ou com envio previsto fora do prazo são descartados imediatamente.

\subsection{Pseudocódigo Simplificado}

\begin{verbatim}
while true:
  now = time_ms()
  pkt = pq.peek()    # Ordenado por deadline
  if not pkt: sleep(short)
  else:
      if pkt.enqueue_time + pkt.deadline < now:
          pq.pop()  # Expirou
          continue
      if estimated_tx_time > pkt.deadline:
          pq.pop()  # Não vai cumprir deadline
          continue
      send(pkt)
\end{verbatim}

\section{Métricas e Metodologia de Avaliação}

\begin{itemize}[noitemsep]
    \item Percentagem de pacotes entregues dentro do deadline (\% on-time).
    \item Latência média e p95/p99 por prioridade.
    \item Goodput útil (bytes entregues a tempo por segundo).
    \item Overhead de controlo (ACK/FEC).
\end{itemize}

\textbf{Cenários de teste:}
\begin{itemize}[noitemsep]
    \item Carga baixa vs carga elevada.
    \item Fluxos mistos: tráfego crítico + tráfego bulk.
    \item Perdas aleatórias vs bursts.
    \item Estudo de parâmetros (tamanho da fila, intensidade de pacing).
\end{itemize}

\section{Implementação e Resultados (a preencher)}

Esta secção será completada após o desenvolvimento do protótipo do DTP.  
Deve incluir:

\begin{itemize}[noitemsep]
    \item Descrição da implementação (linguagem, bibliotecas).
    \item Topologia de testes.
    \item Cenários e métricas recolhidas.
    \item Gráficos e comparação com baseline (UDP / QUIC-lite).
\end{itemize}

\section{Conclusões}

O DTP oferece uma abordagem prática para transportar dados sensíveis ao tempo, permitindo que a própria aplicação defina deadlines e prioridades. Isto permite reduzir latência, eliminar o envio de dados inúteis e melhorar a experiência em aplicações tempo-real. O protocolo é extensível e pode futuramente integrar FEC adaptativo e coexistência com QUIC.

\section{Referências}

\small
\begin{itemize}[leftmargin=10pt]
    \item Shi, Y., et al. (2019). Deadline-Aware Transport. APNet.
    \item RFC 6363 — Forward Error Correction Framework.
    \item RFCs e drafts relacionados com QUIC e HTTP/3.
    \item Papers sobre CoDel e FQ-CoDel.
\end{itemize}

\end{document}
